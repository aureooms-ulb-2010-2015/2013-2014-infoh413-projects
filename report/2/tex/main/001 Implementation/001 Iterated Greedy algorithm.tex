\subsection{Iterated Greedy algorithm}
\label{impl:ig}

\subsubsection{Outline}

This algorithm is an hybrid SLS method adpated from the \emph{IG} algorithm found in \cite{DBLP:journals/eor/RuizS08}.

The outline of the algorithm can be seen in \ref{app:alg:ig}.

The idea is to randomly destroy and then greedily reconstruct the solution.
After the reconstruction, a Local Search is applied and finally the obtained \emph{LO} is eiter accepted or rejected following a \emph{SA} \emph{look-alike} acceptance criterion (see the metropolis condition detailed in \ref{impl:ig:metropolis}).


\subsubsection{Initial solution}

The \emph{NEH} construction heuristics used in \cite{DBLP:journals/eor/RuizS08} have not been used. Instead the two options of implementation exercise 1 are available : slack heuristic or random permutation.


\subsubsection{Local search}

The local search steps are achieved through the simple \emph{II} algorithm implemented in exercise 1 (using a random walk on the insert neighborhood and the \emph{first improvement} pivoting method).


\subsubsection{Termination criterion}

The termination criterion can be specified eiter by a maximum number of steps or a maximum amount of computation time.

\subsubsection{Destruction sample size}

The sample size $d$ is left as a parameter for the user of the program.


\subsubsection{Metropolis condition}
\label{impl:ig:metropolis}

In this version, the metropolis condition is computed based on the relative difference between the two solutions $\pi'$ and $\pi$, i.e. $\frac{\operatorname{O}(\pi') - \operatorname{O}(\pi)}{\operatorname{O}(\pi)}$ in order to allow a better scaling of the parameter set with the current \emph{TWT} of a specific instance. The parameter $T$ is not computed as in \cite{DBLP:journals/eor/RuizS08}, a more \emph{SA}-like approach is chosen here, i.e. $T$ is determined by parameters $T_d$ and $T_p$. The parameters $T_d$ and $T_p$ must be interpreted as follows : the relative deviation $T_d$ has probability $T_p$ to be accepted by the metropolis condition. And from

$$ T_p = \exp{-\frac{T_d}{T}} $$

follows

$$ T = - \frac{T_d}{\log{T_p}} $$